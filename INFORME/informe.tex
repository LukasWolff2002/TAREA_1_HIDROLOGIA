%NO MODIFICAR ESTA SECCION!
\documentclass{article} % Define la clase del documento, en este caso, un artículo
\usepackage[letterpaper,margin=3cm]{geometry} % Configura el tamaño del papel y los márgenes del documento
\usepackage{graphicx} % Permite la inserción de imágenes
\usepackage[spanish]{babel}% Activar esta configuración para informes en español, ajusta el idioma del documento
\usepackage[usenames]{color} % Permite el uso de colores definidos por nombre en el documento
\usepackage{hyperref} % Habilita enlaces y referencias dentro del documento
\hypersetup{colorlinks=true, linkcolor = black, citecolor= black} % Configura el color de los enlaces y citas
\usepackage{booktabs} % Proporciona comandos para crear tablas de alta calidad
\usepackage{natbib} % Permite el uso de citas y referencias bibliográficas con diferentes estilos
\usepackage{tikz} % Permite la creación de gráficos y diagramas vectoriales directamente en LaTeX
\usepackage{float} % Para controlar la posición de los elementos flotantes, como imágenes, con la opción [H]
\bibliographystyle{agsm} % Define el estilo de citas y bibliografía (en este caso, el estilo AGSM)
\usepackage{diagbox} % Permite crear celdas con líneas diagonales en tablas
\usepackage{listings} % Permite la inclusión y formateo de código fuente en el documento
\usepackage{xcolor} % Paquete para definir y usar colores en el documento
\usepackage{parskip} % Añade espacio entre párrafos en lugar de sangrías
\usepackage{fancyhdr} % Permite personalizar encabezados y pies de página
\usepackage{amsmath} % Proporciona una amplia variedad de entornos y comandos matemáticos

\pagestyle{fancy} % Usa el estilo fancyhdr
\fancyhf{} % Borra todos los encabezados y pies de página
\renewcommand{\headrulewidth}{0pt}
\renewcommand{\footrulewidth}{0pt} % Desactiva la línea horizontal predeterminada en el pie
\setlength{\headheight}{2cm} % Ajusta la altura del encabezado para hacer espacio para la línea
\fancyhead[L]{\raisebox{0.20cm}{\textbf{Hidrología}}} % Añade el texto en la parte izquierda del encabezado, subiéndolo ligeramente
\fancyhead[R]{\raisebox{0.1cm}{\includegraphics[width=0.25\linewidth]{LOGO_UNIVERSIDAD.jpg}}} % Añade la imagen en la parte derecha del encabezado y súbela un poco
\fancyhead[C]{\rule{\textwidth}{0.6pt}} % Añade una línea horizontal superior centrada
\fancyfoot[C]{\rule{\textwidth}{0.6pt}} % Añade una línea horizontal en el pie de página centrada
\fancyfoot[R]{\raisebox{-1.5\baselineskip}{\thepage}} % Coloca el número de página a la derecha, con suficiente espacio debajo de la línea
\geometry{top=3cm, bottom=2.5cm} % Ajusta los márgenes superior e inferior

% Definición de colores al estilo Visual Studio Code
\definecolor{codegreen}{rgb}{0.25,0.49,0.48} % Comentarios
\definecolor{codegray}{rgb}{0.5,0.5,0.5} % Números y anotaciones
\definecolor{codepurple}{rgb}{0.58,0,0.82} % Palabras clave
\definecolor{backcolour}{rgb}{0.95,0.95,0.92} % Color de fondo

% Configuración del estilo de las celdas de código
\lstset{
    backgroundcolor=\color{backcolour},   % color de fondo; necesita que el paquete color o xcolor esté cargado
    commentstyle=\color{codegreen},       % estilo de comentarios
    keywordstyle=\color{codepurple},      % estilo de palabras clave
    numberstyle=\tiny\color{codegray},    % estilo de los números de línea
    stringstyle=\color{red},              % estilo de las cadenas de texto
    basicstyle=\ttfamily\small,           % estilo del texto básico
    breakatwhitespace=false,              % ajustes de líneas sólo en espacios en blanco
    breaklines=true,                      % ajustar las líneas si son muy largas
    captionpos=b,                         % posición de la leyenda (abajo)
    keepspaces=true,                      % preserva los espacios en el texto; útil si se usa monoespaciado
    numbers=left,                         % dónde poner los números de línea
    numbersep=5pt,                        % qué tan lejos están los números de línea del código
    showspaces=false,                     % mostrar espacios con subrayados particulares; reemplaza 'showstringspaces'
    showstringspaces=false,               % subrayar los espacios dentro de las cadenas solo
    showtabs=false,                       % mostrar tabulaciones en el código con subrayados particulares
    tabsize=2,                            % tamaños de tabulación a 2 espacios
    language=TeX,                         % lenguaje del código
    morecomment=[l]\#,                    % reconocer # como inicio de comentario en Python
    frame=single,                         % agregar un marco simple alrededor del código
    rulecolor=\color{black}               % color del marco
}

\begin{document}
%----------------------------------------------------------------------------------------
%   PORTADA
%Modificar desde aqui en adelante
%----------------------------------------------------------------------------------------
\begin{titlepage}%Inicio de la carátula, solo modificar los datos necesarios
\newcommand{\HRule}{\rule{\linewidth}{0.5mm}} 
\center 
%----------------------------------------------------------------------------------------
%	ENCABEZADO
%----------------------------------------------------------------------------------------
\includegraphics[width=10cm]{LOGO_UNIVERSIDAD.jpg}\\ % Si esta plantilla se copio correctamente, va a llevar la imagen del logo de la facultad.OBS: Es necesario incluir el paquete: graphicx
\vspace{3cm}
%----------------------------------------------------------------------------------------
%	SECCION DEL TITULO
%----------------------------------------------------------------------------------------
\HRule \\[0.4cm]
{ \huge \bfseries Tarea 1}\\[0.4cm] % Titulo del documento
{ \huge \bfseries Hidrología}\\[0.4cm] % Titulo del documento
\HRule \\[1.5cm]
 \vspace{5cm}
%----------------------------------------------------------------------------------------
%	SECCION DEL AUTOR
%----------------------------------------------------------------------------------------
\begin{flushright}
    { \textbf{Profesor:}\\
    Ricardo Gonzales\\
    \vspace{0.2cm}
    \textbf{Alumnos:}\\
    Bernardo Caprile Canala-Echevarría\\
    Pedro Tomás Valenzuela Bejares\\
    Felipe Alberto Vicencio Fossa \\
    Lukas Wolff Casanova\\
    \vspace{0.2cm}

}
\end{flushright}
\vspace{1cm}
%----------------------------------------------------------------------------------------
%	SECCION DE LA FECHA
%----------------------------------------------------------------------------------------
{\large \textbf{\today}}\\[2cm] % El comando \today coloca la fecha del dia, y esto se actualiza con cada compilacion, en caso de querer tener una fecha estatica, reemplazar el \today por la fecha deseada
\end{titlepage}
%----------------------------------------------------------------------------------------
%  INDICE
%----------------------------------------------------------------------------------------
\newpage
\tableofcontents
\thispagestyle{plain} % Deshabilita el encabezado en la página del índice
\thispagestyle{empty} % Deshabilita el número de página en la página del índice
\newpage

%Se puede agregar un indice de figuras si es nesesario
%\newpage
%\listoffigures 
%\thispagestyle{plain} % Deshabilita el encabezado en la página del índice %
%\thispagestyle{empty}
%\newpage
%----------------------------------------------------------------------------------------
%   ACÁ EMPIEZA EL INFORME
\setcounter{page}{1} % Reinicia el contador de páginas
%----------------------------------------------------------------------------------------
%Este es el formato a seguir para los titulos de las secciones

\section{Introducción}

Este informe tiene como objetivo presentar los resultados obtenidos para la tarea de hidrología asignada, utilizando los datos de la cuenca del Río Mapocho en Los Almendros (código DGA 5722002). Los análisis realizados incluyen la identificación de parámetros representativos de la cuenca, el cálculo de series de caudales, y la estimación de la evapotranspiración media anual. Finalmente, se realiza un balance hidrológico para determinar el caudal promedio anual disponible para abastecimiento de agua desde un embalse planeado en la cuenca.

El análisis se ha realizado utilizando datos obtenidos de fuentes confiables como el portal de la DGA y el Explorador de Cuencas del CR2, para asegurar la precisión de los cálculos y las conclusiones.

\newpage
\section{Pregunta 1}

\subsection{Parámetros representativos de la cuenca}

\begin{table}[h!]
\centering
\begin{tabular}{>{\raggedright}p{6cm} p{8cm}}
\toprule
\textbf{Parámetro} & \textbf{Valor} \\
\midrule
Código DGA & 5722002 \\
Nombre de la estación & Río Mapocho en Los Almendros \\
Coordenadas geográficas del punto de salida & Latitud: -33.37, Longitud: -70.45 \\
Coordenadas UTM del punto de salida & (Por determinar) \\
Elevación del punto de interés & 968 m s.n.m. \\
Área aportante & 638 km² \\
Elevación media de la cuenca & 2779 m s.n.m. \\
Elevación máxima de la cuenca & 5431 m s.n.m. \\
Trazado de la cuenca aportante & \textit{(Delimitación de la cuenca usando herramientas GIS)} \\
\bottomrule
\end{tabular}
\caption{Parámetros representativos de la cuenca del Río Mapocho en Los Almendros}
\label{table:parameters}
\end{table}

La cuenca 5722002 del Río Mapocho en Los Almendros, situada en la región central de Chile, tiene un área de 638 km². La precipitación anual media en esta cuenca es de 501 mm, según datos del CR2MET. La cuenca presenta un índice de aridez de 1.4 y varía en altitud desde los 968 m s.n.m. en el punto de salida hasta los 5431 m s.n.m. en su cota máxima, con una altitud media de 2779 m s.n.m.

\begin{itemize}
    \item \textbf{Zona 1:}
    \begin{itemize}
        \item \textbf{Código estación:} 5722002
        \item \textbf{Nombre estación:} Río Mapocho En Los Almendros
        \item \textbf{Ubicación:} Lat. -33.37, Lon. -70.45
        \item \textbf{Comienzo de observaciones:} 1999-08-01
        \item \textbf{Término de observaciones:} 2019-12-31
    \end{itemize}

    \item \textbf{Zona 2:}
    \begin{itemize}
        \item \textbf{Código estación:} 5720001
        \item \textbf{Nombre estación:} Río Molina Antes Junta San Francisco
        \item \textbf{Ubicación:} Lat. -33.37, Lon. -70.4
        \item \textbf{Comienzo de observaciones:} 2009-11-01
        \item \textbf{Término de observaciones:} 2019-12-31
    \end{itemize}

    \item \textbf{Zona 3:}
    \begin{itemize}
        \item \textbf{Código estación:} 5720003
        \item \textbf{Nombre estación:} La Ermita Central En Bocatoma
        \item \textbf{Ubicación:} Lat. -33.34, Lon. -70.36
        \item \textbf{Comienzo de observaciones:} 1987-05-01
        \item \textbf{Término de observaciones:} 2011-01-31
    \end{itemize}

    \item \textbf{Zona 4:}
    \begin{itemize}
        \item \textbf{Código estación:} 5721016
        \item \textbf{Nombre estación:} Río San Francisco Antes Junta Estero Yerba Loca
        \item \textbf{Ubicación:} Lat. -33.31, Lon. -70.36
        \item \textbf{Comienzo de observaciones:} 2013-04-01
        \item \textbf{Término de observaciones:} 2019-08-30
    \end{itemize}

    \item \textbf{Zona 5:}
    \begin{itemize}
        \item \textbf{Código estación:} 5721017
        \item \textbf{Nombre estación:} Estero Yerba Loca En Piedra Carvajal
        \item \textbf{Ubicación:} Lat. -33.22, Lon. -70.27
        \item \textbf{Comienzo de observaciones:} 2013-09-01
        \item \textbf{Término de observaciones:} 2019-12-31
    \end{itemize}
\end{itemize}

Estas estaciones han sido clave para obtener datos meteorológicos relevantes en distintas áreas de la cuenca, proporcionando información valiosa para la gestión del agua y la investigación ambiental.

\newpage
\subsection{Información de Caudales Medios}

\subsubsection{Caudales Medios Mensuales}

Se considerará un periodo de 30 años, comenzando el 1 de abril de 1993 y finalizando el 31 de marzo de 2023. El año hidrológico empieza en abril y termina en marzo. Los datos se obtuvieron de \url{https://snia.mop.gob.cl/BNAConsultas/reportes}.

A continuación se presentan los caudales medios mensuales por estación a la salida de la cuenca (Zona 1):

\begin{table}[ht]
\centering
\caption{Caudales Medios Mensuales}
\vspace{0.2cm}
\resizebox{\textwidth}{!}{%
\begin{tabular}{|c|c|c|c|c|c|c|c|c|c|c|c|c|}
\hline
Año & Enero & Febrero & Marzo & Abril & Mayo & Junio & Julio & Agosto & Septiembre & Octubre & Noviembre & Diciembre \\
\hline
1993 & - & - & - & 3.46 & 18.46 & 6.87 & 4.93 & 5.10 & 5.44 & 6.45 & 6.72 & 7.20 \\
1994 & 6.19 & 3.56 & 2.62 & 1.97 & 1.45 & 1.46 & 3.51 & 4.36 & 5.34 & 6.25 & 7.12 & 6.45 \\
1995 & 5.26 & 3.11 & 2.14 & 1.84 & 1.58 & 2.30 & 2.20 & 2.72 & 4.76 & 4.77 & 5.73 & 4.47 \\
1996 & 2.85 & 2.05 & 1.67 & 1.21 & 0.70 & 0.64 & 0.70 & 0.81 & 0.80 & 1.35 & 1.46 & 1.66 \\
1997 & 2.32 & 1.71 & 1.44 & 0.96 & 1.36 & 12.47 & 6.21 & 10.41 & 16.96 & 17.75 & 22.07 & 19.74 \\
1998 & 13.33 & 6.21 & 3.75 & 2.81 & 2.10 & 1.68 & 1.30 & 0.92 & 0.81 & 1.24 & 1.58 & 2.76 \\
1999 & 2.60 & 2.58 & 1.54 & 1.07 & 0.77 & 0.70 & 1.04 & 1.65 & 8.57 & 6.94 & 7.48 & 5.69 \\
2000 & 4.16 & 2.63 & 1.58 & 0.77 & 1.02 & 5.15 & 9.25 & 8.29 & 10.63 & 18.72 & 14.39 & 15.07 \\
2001 & 7.18 & 3.75 & 1.20 & 1.07 & 2.33 & 2.07 & 8.85 & 15.05 & 10.04 & 13.33 & 11.16 & 11.71 \\
2002 & 5.05 & 3.22 & 1.99 & 1.47 & 4.70 & 12.35 & 8.47 & 16.92 & 15.20 & 19.89 & 24.61 & 20.37 \\
2003 & 14.94 & 9.28 & 5.74 & 3.93 & 3.08 & 3.52 & 3.64 & 3.41 & 3.87 & 5.78 & 5.78 & 4.65 \\
2004 & 4.01 & 2.89 & 1.85 & 1.58 & 1.44 & 1.49 & 1.55 & 3.14 & 5.77 & 4.18 & 8.43 & 5.31 \\
2005 & 3.64 & 2.95 & 2.42 & 1.63 & 2.08 & 10.79 & 5.58 & 14.06 & 12.29 & 16.35 & 22.97 & 18.37 \\
2006 & 10.11 & 5.56 & 3.47 & 2.51 & 2.15 & 2.39 & 10.79 & 5.97 & 6.69 & 9.49 & 12.17 & 9.21 \\
2007 & 5.33 & 3.07 & 2.54 & 2.13 & 1.93 & 2.31 & 2.90 & 2.52 & 4.61 & 6.55 & 5.38 & 3.97 \\
2008 & 3.35 & 2.57 & 2.63 & 2.03 & 8.95 & 10.77 & 3.62 & 8.80 & 10.14 & 13.98 & 16.96 & 10.33 \\
2009 & 5.94 & 2.98 & 2.27 & 1.86 & 1.95 & 1.97 & 2.38 & 4.28 & 14.20 & 11.77 & 10.56 & 9.12 \\
2010 & 4.16 & 4.23 & 3.30 & 2.53 & 2.31 & 2.62 & 2.80 & 3.41 & 4.33 & 4.90 & 6.03 & 3.62 \\
2011 & 2.74 & 2.15 & 1.44 & 1.45 & 1.12 & 1.20 & 1.37 & 1.77 & 4.24 & 4.41 & 3.66 & 3.87 \\
2012 & 3.61 & 3.10 & 2.22 & 1.49 & 1.61 & 2.62 & 3.19 & 2.29 & 3.39 & 3.35 & 4.66 & 3.37 \\
2013 & 3.59 & 2.98 & 1.83 & 1.41 & 1.43 & 1.82 & 1.82 & 2.69 & 3.53 & 4.56 & 4.32 & 3.43 \\
2014 & 2.84 & 2.26 & 1.69 & 1.32 & 1.03 & 1.44 & 1.41 & 1.81 & 2.46 & 3.98 & 2.43 & 2.57 \\
2015 & 2.40 & 2.20 & 2.20 & 1.45 & 1.26 & 1.11 & 1.21 & 4.03 & 4.82 & 7.44 & 8.63 & 7.99 \\
2016 & 4.33 & 2.93 & 1.38 & 5.07 & 3.24 & - & 3.27 & 4.33 & 5.58 & 5.66 & 7.38 & 7.62 \\
2017 & 5.56 & 2.84 & 2.00 & 2.32 & 2.34 & 3.05 & 2.65 & 2.70 & 4.75 & 6.00 & 5.39 & 4.57 \\
2018 & 2.96 & 2.82 & 1.60 & 1.16 & 1.02 & 1.39 & 1.51 & 1.61 & 2.57 & 3.05 & 3.78 & 3.28 \\
2019 & 3.06 & 2.04 & 1.63 & 1.12 & 0.86 & 1.14 & 1.16 & 1.10 & 1.37 & 1.17 & 1.41 & 1.77 \\
2020 & 2.26 & 1.51 & 1.08 & 0.86 & 0.69 & 0.89 & 1.51 & 2.04 & 3.34 & 3.58 & 2.84 & 3.34 \\
2021 & 3.01 & 3.21 & 1.99 & 1.04 & 0.96 & 0.89 & 0.84 & 1.35 & 2.40 & 2.19 & 1.67 & 1.84 \\
2022 & 1.69 & 1.06 & 1.03 & 1.00 & 1.06 & 0.81 & 0.89 & 1.56 & 1.81 & 2.22 & - & - \\
2023 & - & - & - & - & - & - & - & - & - & - & - & - \\
\hline
\end{tabular}%
}
\label{tab:caudales_mensuales}
\end{table}

\newpage
\subsubsection{Caudales Medios Temporada}

A continuación se presentan los caudales promedio por temporada, calculados a partir de los datos mensuales. La temporada se define como el período de un año hidrológico desde abril hasta marzo del año siguiente. Los valores reflejan el caudal medio durante cada temporada hidrológica desde 1993 hasta 2023.

\begin{table}[h!]
\centering
\small % Reduce el tamaño de la fuente de la tabla
\caption{Caudales Medios por Temporada}
\begin{tabular}{|c|c|}
\hline
\textbf{Temporada} & \textbf{Caudal Promedio (m³/s)} \\
\hline
1993-1994 & 6.41666667 \\
1994-1995 & 4.035 \\
1995-1996 & 3.07833333 \\
1996-1997 & 1.23333333 \\
1997-1998 & 10.935 \\
1998-1999 & 1.82666667 \\
1999-2000 & 3.52333333 \\
2000-2001 & 7.95166667 \\
2001-2002 & 7.15583333 \\
2002-2003 & 12.8283333 \\
2003-2004 & 3.8675 \\
2004-2005 & 3.49166667 \\
2005-2006 & 10.2716667 \\
2006-2007 & 6.02583333 \\
2007-2008 & 3.40416667 \\
2008-2009 & 8.06416667 \\
2009-2010 & 5.815 \\
2010-2011 & 3.24 \\
2011-2012 & 2.66833333 \\
2012-2013 & 2.86416667 \\
2013-2014 & 2.65 \\
2014-2015 & 2.10416667 \\
2015-2016 & 3.88166667 \\
2016-2017 & 4.77727273 \\
2017-2018 & 3.42916667 \\
2018-2019 & 2.175 \\
2019-2020 & 1.32916667 \\
2020-2021 & 2.275 \\
2021-2022 & 1.41333333 \\
2022-2023 & 1.33571429 \\
\hline
\end{tabular}
\label{tab:caudales_temporada}
\end{table}

\newpage
\subsection{Información de Precipitaciones Medias}
A continuación se presentan las precipitaciones totales por cada estación mensualmente, de esta forma, es posible obtener la precipitación media mensual. 

Para obtener las precipitaciones, se utilizó la siguiente base de datos: \url{https://snia.mop.gob.cl/BNAConsultas/reportes}

\subsubsection{Estación Río Mapocho En Los Almendros}

\begin{table}[ht]
\centering
\caption{Precipitaciones Estación Río Mapocho En Los Almendros}
\resizebox{\textwidth}{!}{%
\begin{tabular}{|c|c|c|c|c|c|c|c|c|c|c|c|c|}
\hline
Año & Enero & Febrero & Marzo & Abril & Mayo & Junio & Julio & Agosto & Septiembre & Octubre & Noviembre & Diciembre \\
\hline
1999 & - & - & - & 0.00 & 0.00 & 0.00 & 122.10 & 84.70 & 24.40 & 0.00 & 0.00 & 0.00 \\
2000 & 9.80 & 0.00 & 37.60 & 30.40 & 308.50 & 76.90 & 14.60 & 123.50 & 10.60 & 15.30 & 0.00 & 0.00 \\
2001 & 0.00 & 0.00 & 15.70 & 27.10 & 68.40 & 10.50 & 186.20 & 78.80 & 48.90 & 21.10 & 0.70 & 0.00 \\
2002 & 1.30 & 0.00 & 32.20 & 38.30 & 158.70 & 193.10 & 96.20 & 114.80 & 45.00 & 16.40 & 3.80 & 1.70 \\
2003 & 11.30 & 0.00 & 0.30 & 30.80 & 23.20 & 76.70 & 7.50 & 18.20 & 0.00 & 12.10 & 0.00 & 0.00 \\
2004 & 1.50 & 20.10 & 40.90 & 20.50 & 55.70 & 88.60 & 73.10 & 68.20 & 20.50 & 98.90 & 0.00 & 0.00 \\
2005 & 8.50 & 0.00 & 16.60 & 7.30 & 76.60 & 148.70 & 41.00 & 227.70 & 53.10 & 64.00 & 11.90 & 0.00 \\
2006 & 0.00 & 0.00 & 0.00 & 0.80 & 5.00 & 74.10 & 219.90 & 46.90 & 10.00 & 58.90 & 2.20 & 0.00 \\
2007 & 0.00 & 41.70 & 9.10 & 6.40 & 89.00 & 60.00 & 53.00 & 0.50 & 0.00 & 0.00 & 0.00 & 0.00 \\
2008 & 0.00 & 16.00 & 5.80 & 163.80 & 62.40 & 31.90 & 164.50 & 9.50 & 0.30 & 0.00 & 0.00 & 0.00 \\
2009 & 0.00 & 3.00 & 0.50 & 0.00 & 10.80 & 106.40 & 41.60 & 115.40 & 101.50 & 20.60 & 0.00 & 0.00 \\
2010 & 0.00 & 0.00 & 0.00 & 0.80 & 59.30 & 90.80 & 44.80 & 3.40 & 39.30 & 20.50 & 69.80 & 0.80 \\
2011 & 0.00 & 3.70 & 0.00 & 0.00 & 49.10 & 56.70 & 37.20 & 13.30 & 17.20 & 2.40 & 0.00 & 0.00 \\
2012 & 0.00 & 0.00 & 38.30 & 39.80 & 73.80 & 8.40 & 61.40 & 5.00 & 54.00 & 5.80 & 38.00 & 0.00 \\
2013 & 1.10 & 1.60 & 0.00 & 0.00 & 73.80 & 24.60 & 9.00 & 34.40 & 17.20 & 0.20 & 0.00 & 0.00 \\
2014 & 0.40 & 0.10 & 0.00 & 0.40 & 15.00 & 73.60 & 25.20 & 50.80 & 33.40 & 0.40 & 8.40 & 0.00 \\
2015 & 0.00 & 3.20 & 22.20 & 0.20 & 0.00 & 31.60 & 84.00 & 43.80 & 57.40 & 9.00 & 0.00 & 0.00 \\
2016 & 0.00 & 0.00 & 107.60 & 31.00 & 48.00 & 0.00 & 0.40 & 0.00 & 0.00 & 0.00 & 0.00 & 0.00 \\
2017 & 0.00 & 0.00 & 0.00 & 7.20 & 56.60 & 1.80 & 0.00 & 0.00 & 0.00 & 0.00 & 0.00 & 0.00 \\
2018 & 0.00 & 0.00 & 1.40 & 0.00 & 1.00 & 29.40 & 21.60 & 30.40 & 0.00 & 0.00 & 0.00 & 0.00 \\
2019 & 0.00 & 0.00 & 0.00 & 0.00 & 6.80 & 20.00 & 22.00 & 9.00 & 0.20 & 0.00 & 0.00 & 0.00 \\
2020 & 0.00 & 0.00 & 0.00 & 1.80 & 5.80 & 58.60 & 70.00 & 15.80 & 0.40 & 0.00 & 0.00 & 0.00 \\
2021 & - & - & - & - & - & - & - & - & - & - & - & - \\
2022 & - & - & - & - & - & - & - & - & - & - & - & - \\
2023 & - & - & - & - & - & - & - & - & - & - & - & - \\
\hline
\end{tabular}%
}
\label{tab:precipitaciones_rio_mapocho}
\end{table}


\subsubsection{Estación Río Molina Antes de junta con San Francisco}

\begin{table}[ht]
\centering
\caption{Precipitaciones Estación Río Molina Antes Junta San Francisco}
\resizebox{\textwidth}{!}{%
\begin{tabular}{|c|c|c|c|c|c|c|c|c|c|c|c|c|}
\hline
Año & Enero & Febrero & Marzo & Abril & Mayo & Junio & Julio & Agosto & Septiembre & Octubre & Noviembre & Diciembre \\
\hline
2009 & - & - & - & - & - & - & - & - & - & - & 0.9 & 0.0 \\
2010 & 0.0 & 0.0 & 0.0 & 1.7 & 68.5 & 109.3 & 26.1 & 1.5 & 36.2 & 20.4 & 60.7 & 2.9 \\
2011 & 0.1 & 12.5 & - & 2.4 & 0.0 & 29.4 & 59.0 & 37.8 & 13.9 & 17.1 & 2.5 & 0.0 \\
2012 & 5.4 & 0.0 & 0.0 & 42.7 & 50.2 & 91.3 & 7.5 & 65.6 & 4.9 & 52.6 & 5.5 & 34.7 \\
2013 & 2.8 & 8.4 & 0.0 & 1.0 & 119.4 & 47.6 & 16.7 & 56.7 & 24.9 & 3.2 & 0.0 & 0.2 \\
2014 & 0.4 & 1.9 & 2.6 & 1.0 & 19.3 & 97.5 & 26.9 & 72.9 & 37.1 & 0.3 & 10.0 & 0.5 \\
2015 & 0.0 & 12.8 & 44.0 & 0.0 & 0.8 & 0.0 & 55.9 & 132.5 & 64.8 & 63.4 & 9.4 & 0.0 \\
2016 & 6.9 & 0.0 & 0.0 & 132.0 & 62.8 & 64.8 & 38.7 & 0.8 & 7.5 & 47.5 & 6.3 & 56.5 \\
2017 & 0.0 & 0.2 & 0.5 & 13.8 & 109.4 & 88.7 & 5.2 & 0.0 & 0.0 & 0.0 & 0.0 & 0.0 \\
2018 & 0.1 & 0.0 & 1.8 & 0.0 & 13.1 & 0.1 & 0.0 & 20.2 & 0.0 & 0.0 & 0.0 & 0.0 \\
2019 & 0.0 & 0.0 & 0.3 & 0.0 & 18.6 & 27.0 & 0.4 & 0.0 & 0.0 & 20.6 & 0.0 & 0.0 \\
2020 & 0.0 & 0.0 & 0.0 & 3.2 & 6.0 & 97.4 & 55.3 & 13.9 & 0.0 & 0.0 & 0.0 & 0.0 \\
2021 & 0.0 & 0.0 & 0.0 & 0.0 & 0.0 & 0.0 & 0.0 & 0.0 & 0.0 & 0.0 & 0.0 & 0.0 \\
2022 & 0.0 & 0.0 & 0.0 & 0.0 & 0.0 & 0.0 & 0.0 & 0.0 & 0.0 & 0.0 & 0.0 & 0.0 \\
2023 & 0.0 & 0.0 & 0.0 & 0.0 & - & - & - & - & - & - & - & - \\
\hline
\end{tabular}%
}
\label{tab:precipitaciones_rio_molina}
\end{table}

\subsubsection{Estación La Ermita Central En Bocatoma}

La estación no presenta registros de precipitaciones

\subsubsection{Estación Río San Francisco Antes de junta con Estero Yerba Loca}

\begin{table}[ht]
\centering
\caption{Precipitaciones Estación Río San Francisco Antes Junta Estero Yerba Loca}
\resizebox{\textwidth}{!}{%
\begin{tabular}{|c|c|c|c|c|c|c|c|c|c|c|c|c|}
\hline
Año & Enero & Febrero & Marzo & Abril & Mayo & Junio & Julio & Agosto & Septiembre & Octubre & Noviembre & Diciembre \\
\hline
2013 & - & - & - & 4.8 & 154.4 & 66.0 & 39.0 & 75.0 & 36.4 & 3.4 & 1.6 & 0.8 \\
2014 & 3.0 & 12.2 & 2.2 & 1.6 & 29.8 & 118.0 & 14.4 & 64.4 & 28.8 & 4.4 & 24.6 & 0.4 \\
2015 & 0.0 & 21.2 & 0.0 & 0.0 & 4.0 & 4.0 & 0.0 & 94.0 & 354.6 & 315.0 & 138.6 & 107.6 \\
2016 & 10.9 & 0.0 & 0.0 & 119.9 & 167.4 & 104.0 & 56.5 & 52.2 & 52.2 & 0.0 & 56.7 & 64.0 \\
2017 & 0.0 & 0.1 & 1.3 & 17.4 & 108.4 & 142.2 & 50.0 & 15.5 & 15.5 & 0.0 & 0.0 & 0.0 \\
2018 & 0.0 & 0.0 & 0.0 & 0.0 & 0.0 & 20.3 & 20.3 & 1.3 & 11.0 & 9.7 & 0.0 & 0.0 \\
2019 & 0.0 & 0.0 & 0.0 & 0.0 & 7.2 & 26.7 & 10.3 & 0.5 & 0.0 & 11.6 & 0.1 & 0.0 \\
2020 & 0.0 & 0.0 & 0.0 & 5.4 & 8.8 & 109.2 & 44.4 & 13.6 & 0.0 & 0.0 & 0.0 & 0.0 \\
\hline
\end{tabular}%
}
\label{tab:precipitaciones_rio_san_francisco}
\end{table}

\subsubsection{Estacion Estero Yerba Loca En Piedra Carvajal}

La toma de datos de esta estación no fue regular, por lo tanto, no se considerará para efectos de cálculo.

\begin{table}[ht]
\centering
\caption{Precipitaciones Estación Estero Yerba Loca en Piedra Carvajal}
\resizebox{\textwidth}{!}{%
\begin{tabular}{|c|c|c|c|c|c|c|c|c|c|c|c|c|}
\hline
Año & Enero & Febrero & Marzo & Abril & Mayo & Junio & Julio & Agosto & Septiembre & Octubre & Noviembre & Diciembre \\
\hline
2011 & - & - & - & 0.0 & 0.0 & 0.0 & 0.0 & 0.0 & 0.0 & 0.0 & 0.0 & 0.0 \\
2012 & 0.00 & 0.00 & 0.00 & 0.00 & 0.00 & 0.00 & 0.00 & 0.00 & 0.00 & 0.00 & 0.00 & 0.00 \\
2013 & 0.00 & 0.00 & 0.00 & 0.00 & 0.00 & 0.00 & 0.00 & 0.00 & 9.30 & 6.00 & 6.10 & 0.00 \\
2014 & 0.00 & 0.00 & 0.00 & 0.00 & 0.00 & 0.00 & 0.00 & 0.00 & 0.00 & 6.00 & 1.90 & 0.00 \\
2015 & 0.00 & 11.00 & 32.90 & 0.0 & 0.0 & 0.0 & 0.0 & 11.90 & 0.0 & 0.0 & 2.80 & 0.0 \\
2016 & 0.00 & 0.00 & 0.00 & 0.00 & 0.00 & 0.00 & 0.90 & 0.00 & 0.00 & 0.00 & 0.00 & 0.00 \\
2017 & 0.00 & 0.00 & 11.80 & 17.90 & 25.10 & 1.40 & 2.10 & 0.00 & 0.00 & 0.00 & 0.00 & 0.00 \\
2018 & 18.50 & 1.40 & 8.60 & 0.40 & 9.90 & 4.40 & 76.60 & 2.50 & 0.00 & 0.00 & 0.00 & 0.00 \\
2019 & 1.70 & 2.10 & 0.80 & 2.70 & 2.30 & 3.50 & 2.40 & 0.00 & 1.80 & 0.00 & 7.00 & 0.00 \\
2020 & 0.30 & 0.00 & 3.40 & 1.50 & 8.10 & 2.70 & 0.20 & 0.00 & 0.00 & 0.00 & 0.00 & 0.00 \\
2021 & 0.00 & 0.00 & 0.00 & 0.00 & 0.00 & 0.00 & 0.00 & 0.00 & 0.00 & 0.00 & 0.00 & 0.00 \\
2022 & 0.00 & 0.00 & 0.00 & 0.00 & 0.00 & 0.00 & 0.00 & 0.00 & 0.00 & 0.00 & 0.00 & 0.00 \\
2023 & 0.00 & 0.00 & 0.00 & - & - & - & - & - & - & - & - & - \\
\hline
\end{tabular}%
}
\label{tab:precipitaciones_estero_yerba_loca}
\end{table}

\newpage
\subsubsection{Precipitaciones promedio mensuales según las 5 zonas}

\begin{table}[ht]
\centering
\caption{Precipitaciones Promedio Mensuales según las 5 zonas}
\resizebox{\textwidth}{!}{%
\begin{tabular}{|c|c|c|c|c|c|c|c|c|c|c|c|c|}
\hline
Año & Enero & Febrero & Marzo & Abril & Mayo & Junio & Julio & Agosto & Septiembre & Octubre & Noviembre & Diciembre \\
\hline
1993 & - & - & - & - & - & - & - & - & - & - & - & - \\
1994 & - & - & - & - & - & - & - & - & - & - & - & - \\
1995 & - & - & - & - & - & - & - & - & - & - & - & - \\
1996 & - & - & - & - & - & - & - & - & - & - & - & - \\
1997 & - & - & - & - & - & - & - & - & - & - & - & - \\
1998 & - & - & - & - & - & - & - & - & - & - & - & - \\
1999 & 0.00 & 0.00 & 0.00 & 122.10 & 84.70 & 24.40 & 0.00 & 0.00 & 0.00 & - & - & - \\
2000 & 9.80 & 0.00 & 37.60 & 30.40 & 308.50 & 76.90 & 14.60 & 123.50 & 10.60 & 15.30 & 0.00 & 0.00 \\
2001 & 0.00 & 0.00 & 15.70 & 27.10 & 68.40 & 10.50 & 186.20 & 78.80 & 48.90 & 21.10 & 0.70 & 0.00 \\
2002 & 1.30 & 0.00 & 32.20 & 38.30 & 158.70 & 193.10 & 96.20 & 114.80 & 45.00 & 16.40 & 3.80 & 1.70 \\
2003 & 11.30 & 0.00 & 0.30 & 30.80 & 23.20 & 76.70 & 7.50 & 18.20 & 0.00 & 12.10 & 0.00 & 0.00 \\
2004 & 1.50 & 20.10 & 40.90 & 20.50 & 55.70 & 88.60 & 73.10 & 68.20 & 20.50 & 98.90 & 0.00 & 0.00 \\
2005 & 8.50 & 0.00 & 16.60 & 7.30 & 76.60 & 148.70 & 41.00 & 227.70 & 53.10 & 64.00 & 11.90 & 0.00 \\
2006 & 0.00 & 0.00 & 0.00 & 0.80 & 5.00 & 74.10 & 219.90 & 46.90 & 10.00 & 58.90 & 2.20 & 0.00 \\
2007 & 0.00 & 41.70 & 9.10 & 6.40 & 89.00 & 60.00 & 53.00 & 0.50 & 0.00 & 0.00 & 0.00 & 0.00 \\
2008 & 0.00 & 16.00 & 5.80 & 163.80 & 62.40 & 31.90 & 164.50 & 9.50 & 0.30 & 0.00 & 0.00 & 0.00 \\
2009 & 0.00 & 3.00 & 0.50 & 0.00 & 10.80 & 106.40 & 41.60 & 115.40 & 101.50 & 20.60 & 0.00 & 0.00 \\
2010 & 0.00 & 0.00 & 0.00 & 0.80 & 59.30 & 90.80 & 44.80 & 3.40 & 39.30 & 20.50 & 69.80 & 0.80 \\
2011 & 0.00 & 3.70 & 0.00 & 0.00 & 24.55 & 56.70 & 48.10 & 13.30 & 17.20 & 2.40 & 0.00 & 0.00 \\
2012 & 0.00 & 0.00 & 19.15 & 39.80 & 73.80 & 8.40 & 61.40 & 5.00 & 54.00 & 5.80 & 38.00 & 0.00 \\
2013 & 1.10 & 1.60 & 0.00 & 0.50 & 73.80 & 45.30 & 24.00 & 54.70 & 17.20 & 0.20 & 0.00 & 0.00 \\
2014 & 1.70 & 0.10 & 0.00 & 0.70 & 15.00 & 95.80 & 25.20 & 50.80 & 33.40 & 0.40 & 9.20 & 0.00 \\
2015 & 0.00 & 3.20 & 22.07 & 0.07 & 2.00 & 11.87 & 42.00 & 68.90 & 57.40 & 162.00 & 0.00 & 0.00 \\
2016 & 0.00 & 0.00 & 35.87 & 81.50 & 48.00 & 52.00 & 0.40 & 0.00 & 0.00 & 0.00 & 0.00 & 32.00 \\
2017 & 0.00 & 0.00 & 0.00 & 7.20 & 56.60 & 1.80 & 25.00 & 0.00 & 0.00 & 0.00 & 0.00 & 0.00 \\
2018 & 0.00 & 0.00 & 0.70 & 0.00 & 0.50 & 29.40 & 10.80 & 30.40 & 3.67 & 0.00 & 0.00 & 0.00 \\
2019 & 0.00 & 0.00 & 0.00 & 0.00 & 6.80 & 23.50 & 22.00 & 4.50 & 0.07 & 0.00 & 0.00 & 0.00 \\
2020 & 0.00 & 0.00 & 0.00 & 1.80 & 5.90 & 58.60 & 70.00 & 15.80 & 0.20 & 0.00 & 0.00 & 0.00 \\
2021 & - & - & - & - & - & - & - & - & - & - & - & - \\
2022 & - & - & - & - & - & - & - & - & - & - & - & - \\
2023 & - & - & - & - & - & - & - & - & - & - & - & - \\
\hline
\end{tabular}%
}
\label{tab:precipitaciones_promedio}
\end{table}

Los datos de precipitaciones promedio mensuales son aplicables en diferentes zonas según los períodos especificados: Zona 1 cubre desde abril de 1993 hasta septiembre de 2022; Zona 2, desde enero de 2010 hasta agosto de 2020; y Zona 4, desde abril de 2013 hasta agosto de 2019. No hay datos aplicables para las zonas 3 y 5 en los períodos considerados. Estos datos permiten analizar las variaciones y patrones de precipitación en las áreas y tiempos indicados.

\newpage
\subsubsection{Precipitaciones promedio anuales según las 5 zonas}

\begin{table}[ht]
\centering
\caption{Precipitaciones Promedio Anuales (Según Zonas)}
\begin{tabular}{|c|c|}
\hline
\textbf{Periodo} & \textbf{Precipitaciones (mm)} \\
\hline
1993-1994 & 0.00 \\
1994-1995 & 0.00 \\
1995-1996 & 0.00 \\
1996-1997 & 0.00 \\
1997-1998 & 0.00 \\
1998-1999 & 0.00 \\
1999-2000 & 278.60 \\
2000-2001 & 595.50 \\
2001-2002 & 475.20 \\
2002-2003 & 679.60 \\
2003-2004 & 231.00 \\
2004-2005 & 450.60 \\
2005-2006 & 630.30 \\
2006-2007 & 468.60 \\
2007-2008 & 230.70 \\
2008-2009 & 435.90 \\
2009-2010 & 396.30 \\
2010-2011 & 333.20 \\
2011-2012 & 181.40 \\
2012-2013 & 288.90 \\
2013-2014 & 217.50 \\
2014-2015 & 255.77 \\
2015-2016 & 380.10 \\
2016-2017 & 213.90 \\
2017-2018 & 91.30 \\
2018-2019 & 74.77 \\
2019-2020 & 56.87 \\
2020-2021 & 152.30 \\
2021-2022 & 0.00 \\
2022-2023 & 0.00 \\
\hline
\end{tabular}
\label{tab:precipitaciones_anuales}
\end{table}

Entre 1993 y 1998, no se registraron precipitaciones, posiblemente debido a sequías o falta de datos. Desde 1999, las precipitaciones aumentaron notablemente, con picos en 2002-2003 y 2005-2006. Sin embargo, desde 2018 hasta 2023, las precipitaciones han sido muy bajas o inexistentes. Estos patrones sugieren posibles cambios climáticos o sequías prolongadas. Es crucial analizar estos datos junto con otros indicadores climáticos para entender las causas y sus implicaciones para la gestión de los recursos hídricos.

\newpage
\subsection{Balance Hidrológico}

El balance hidrológico se describe matemáticamente con la siguiente ecuación:

\[
\frac{dS}{dt} = X - Y
\]

donde:
\begin{itemize}
    \item $X$ representa las entradas de agua al sistema (como la precipitación, P).
    \item $Y$ representa las salidas de agua del sistema (como la evaporación, E, y el caudal, Q).
    \item $\frac{dS}{dt}$ es la tasa de variación del almacenamiento de agua en el sistema con respecto al tiempo.
\end{itemize}

\subsubsection{Simplificación de la Ecuación}

La ecuación del balance hidrológico puede simplificarse en un contexto de largo plazo o cuando se consideran promedios anuales, donde el cambio en el almacenamiento ($\frac{dS}{dt}$) tiende a cero, resultando en:

\[
P = Q + ET
\]

donde:
\begin{itemize}
    \item $P$ es la precipitación anual media (501 mm en este estudio).
    \item $Q$ es el caudal medio que puede ser calculado en términos mensuales, anuales o diarios.
    \item $ET$ es la evapotranspiración, que incluye tanto la evaporación del suelo y cuerpos de agua como la transpiración de las plantas.
\end{itemize}

\subsubsection{Cálculo de la Evapotranspiración}

La evapotranspiración (ET) teórica se puede calcular utilizando la fórmula de Penman-Monteith o métodos simplificados dependiendo de la disponibilidad de datos. En este estudio, se ha considerado:

\[
ET = f(P, T, Q)
\]

donde $T$ es la temperatura media, que se puede obtener mediante el análisis de isolíneas y otros datos meteorológicos.

\subsection{Cálculo de la Evaporación Media Anual y Mensual}

Para calcular la evaporación media anual y mensual, es fundamental analizar los promedios de precipitación y caudal. Se puede desglosar de la siguiente manera:

\subsubsection{Evaporación Media Anual}

Se basa en el promedio de precipitación anual y las condiciones climáticas del área de estudio. Para una precipitación anual media de 501 mm, la evaporación anual puede ser calculada ajustando los modelos de balance hídrico a las características del clima y suelo locales.

\subsubsection{Evaporación Media Mensual}

La evaporación media mensual se calcula dividiendo el total anual entre los meses del año o utilizando datos mensuales específicos si están disponibles. Esta medida es crucial para el manejo mensual del agua, especialmente en áreas con variabilidad estacional significativa en la precipitación y temperatura.

% \begin{figure}[h]
%     \centering
%     \includegraphics[width=\textwidth]{imagenes/aaaa.png}
%     \caption{\centering Climograma Cuenca Río Mapocho en Los Almendros \\ Fuente: \url{https://chi2.ufro.cl/Mapocho/}}
% \end{figure}

\newpage
\section{Pregunta 2}

\subsection{Datos sobre el Cauce y Embalse}

A continuación, se presentan los datos utilizados en el análisis:

\begin{table}[h]
\centering
\caption{Datos sobre el cauce y embalse a analizar}
\begin{tabular}{lr}
\toprule
\textbf{Parámetro} & \textbf{Valor} \\
\midrule
Área total & 420 km\(^2\) \\
Precipitación media anual & 630 mm/año \\
Escurrimiento medio anual & 280 mm/año \\
Área embalse & 13 km\(^2\) \\
Evaporación media en embalse & 990 mm/año \\
\bottomrule
\end{tabular}
\end{table}

\subsection{Supuestos y Ecuaciones Utilizadas}

Para el desarrollo del análisis se consideraron los siguientes supuestos y ecuaciones:

\begin{itemize}
    \item Caudales subterráneos, retención por vegetación y variaciones en almacenamiento se suponen iguales a 0.
    \item El balance hidrológico se expresa como: Entrada = Salida.
\end{itemize}

Las ecuaciones utilizadas son:

\begin{align}
    Q_{\text{afluente}} &= \text{Escurrimiento medio anual} \times \text{Área total} \\
    Q_{\text{ecológico}} &= 0.1 \times Q_{\text{afluente}} \\
    Q_{\text{Pp embalse}} &= \text{Precipitación media anual} \times \text{Área embalse} \\
    Q_{\text{Ev embalse}} &= \text{Evaporación media en embalse} \times \text{Área embalse} \\
    Q_{\text{efluente}} &= Q_{\text{ecológico}} + Q_{\text{abastecimiento}}
\end{align}

Del balance hidrológico, tenemos:

\begin{equation}
    P_{\text{p}} + Q_{\text{afluente}} = E_{\text{v}} + E_{\text{T}} + Q_{\text{efluente}}
\end{equation}

Donde:

\[
E_{\text{T}} = P_{\text{p}} - \text{Escurrimiento medio anual}
\]

Reemplazando en la ecuación anterior:

\begin{equation}
    P_{\text{p}} + Q_{\text{afluente}} - E_{\text{v}} - E_{\text{T}} - Q_{\text{ecológico}} = Q_{\text{abastecimiento}}
\end{equation}

\newpage
\subsection{Resultados Obtenidos}

Los resultados obtenidos de los caudales se presentan en la siguiente tabla:

\begin{table}[h]
\centering
\caption{Resultados obtenidos de caudales}
\begin{tabular}{lr}
\toprule
\textbf{Caudal} & \textbf{Valor (m\(^3\)/s)} \\
\midrule
\(Q_{\text{afluente}}\) & 3.73 \\
\(Q_{\text{ecológico}}\) & 0.373 \\
\(Q_{\text{Pp}}\) & 0.259 \\
\(Q_{\text{Ev}}\) & 0.408 \\
\(E_{\text{T}}\) & \(1.11 \times 10^{-8}\) \\
\(Q_{\text{abastecimiento}}\) & 3.36 \\
\bottomrule
\end{tabular}
\end{table}

\newpage
\section{Conclusión}
En conclusión, el informe presenta un análisis exhaustivo de la cuenca del Río Mapocho en Los Almendros. Se han examinado los parámetros clave de la cuenca, como su tamaño, elevación y precipitación media anual, utilizando datos de estaciones de monitoreo confiables. Los caudales medios mensuales y por temporada se han calculado para evaluar el flujo de agua en diferentes periodos y años, mientras que la información sobre precipitaciones se ha utilizado para entender la variabilidad en la disponibilidad de agua.

El análisis muestra que la cuenca presenta una variabilidad significativa en los caudales y precipitaciones, lo que influye en la cantidad de agua disponible para abastecimiento. Estos resultados son cruciales para planificar el uso del agua y para el diseño de futuros embalses en la región. La información detallada obtenida de las estaciones de monitoreo ayuda a asegurar la precisión en la evaluación de recursos hídricos y proporciona una base sólida para la gestión del agua en la cuenca.

\end{document}